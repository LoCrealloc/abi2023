
\documentclass{article}

\usepackage{graphicx}
\usepackage{array}
\usepackage[left=2cm, right=2cm, top=2cm, bottom=3cm]{geometry}

\hbadness=10001
\hfuzz=4pt
\vfuzz=4pt

\title{Englisch}
\date{Q2 2022/2023}
\author{Paul SK}

\begin{document}
	\pagenumbering{gobble}
	\maketitle
	\newpage

	\pagenumbering{arabic}

	\section{Textanalyse}

	\subsection{Narrative perspective / point of view}

	\subsubsection{Different narrators}

	\paragraph{First person narrator}
	\begin{itemize}
		\item "I", "We", "Our" etc.
		\item better knowledge about the narrator
		\item less knowledge about / insight into other characters
		\item only one (maybe biased) perspective
	\end{itemize}

	\paragraph{Third person limited narrator}
	\begin{itemize}
		\item focused on one ccaracter
		\item same knowledge as first person narrator (insight etc)
		\item deeper experience
		\item distance to other characters
	\end{itemize}

	\paragraph{Third person omniscient narrator}
	\begin{itemize}
		\item freed from one character
		\item freed from time (time jumps, flashbacks)
		\item opinions from both the narrator and the characters
		\item knows more than the characters
		\item god-like view
	\end{itemize}

	\subsubsection{Effects of different narrative perspectives}

	\paragraph{Suspense}
	Sometimes the omniscient third person narrator tells the reader something the character learns later in the story. The reader therefore waits for this point, which creates a spencial tension that makes the reader want to know more about the continuation of the story.

	\subsubsection{Tense}
	The tense of a text is also important for the narrative perspective. A text that is written in the present creates a more immersive feeling for the reader. Together with a first person or third person limited narrator, the reader feels like he is a part of the story.
	A past tense on the other hand brings some sort of distance into the text and enables the narrator to evaluate events that took place in the past and influence the current life of some characters.

	\subsubsection{Focus on main character}
	It also occurs that the first person / third person limited narrator does not focus on teh main character. This is also interesting, as the narrator is able to judge the protagonists actions.

	\subsection{Style and tone}

	\subsubsection{Style}
	\begin{itemize}
		\item the ways in which something is written
		\item also includes choice of words and tone
		\item how is the information presented?
		\item to what extent is the style appropriate for the audience / the arguments?
	\end{itemize}

	\subsubsection{Choice of words / diction}
	\begin{itemize}
		\item formal/casual/informal words
		\item positive / negative connotated word
		\item vague
		\item overly complex
		\item repetition of certain words
	\end{itemize}

	\subsubsection{Tone}
	\begin{itemize}
		\item subjective / objective
		\item logical / emotional
		\item intimate / distant
		\item serious / humorous
		\item long / short / varying sentences
		\item funny, familiar
	\end{itemize}

	\subsection{Stylistic devices}
	\begin{center}
		\def\arraystretch{1.1}
		\begin{tabular}{ | p{8em} | p{12em} | p{12em} | p{12em} | } 
 			\hline
			\textbf{Device} & \textbf{Definition} & \textbf{Example} & \textbf{Effect} \\ 
			\hline
			\hline
			\textbf{Alliteration} & Repeated use of same initial sound & "She sells seashells by the seashore" & Creates emphasis \\
			\hline
			\textbf{Assonance} & Repetition of vowel sounds in words that are close together & The bee buzzed and the boy blew his bee balm. & Adds harmony and rhythm to language \\
			\hline
			\textbf{Metaphor} & Figure in which a word/phrase is applied to something which it is not literally applicable & "Life is a journey" & Adds depth/meaning \\ 
 			\hline
			\textbf{Simile} & Comparsion of two unlike things & "She sings like an angel" & Creates vivid imagery, adds interest \\
			\hline
			\textbf{Hyperbole} & Exaggeration & "I've told you a million times.." & Creates an emotional impact, adds emphasis to point \\
			\hline
			\textbf{Irony} & Intended meaning of a word/phrase is opposite to its literal meaning & "That was a real smart thing to do" & Adds humor/depth by highlighting discrepancy between excpected and actual observations \\
			\hline
			\textbf{Personification} & Objects/abstract ideas are given human traits/abilities & "The wind howled through the trees" & Adds interest \\
			\hline
			\textbf{Symbolism} & Use of symbols to represent ideas/qualities & A dove for peace, a tombstone for death & Adds depth and meaning by utilizing an single object to represent something larger \\
			\hline 
			\textbf{Rhetorical question} & A question asked for effect with no answer expected & "Why do we even bother?" & Engages the reader, adds emphasis to a point \\
			\hline
			\textbf{Onomatopoeia} & A word that sounds like its meaning & Sizzle, buzz, hiss & Adds sound and musicality to language \\
			\hline
			\textbf{Antithesis} & Balancing two contrasting ideas in a parallel construction & "Ask not what your country can do for you, ask what you can do for your country" & Adds emphasis, balance, and rhythm to language \\
			\hline
			\textbf{Chiasm} & Reversal of grammatical structures in successive clauses & "Not only… but also" & Adds symmetry and balance to language \\
			\hline
			\textbf{Euphemism} & Use of mild or indirect language to refer to something taboo or unpleasant & "Passed away" instead of "died" & Softens the impact of harsh or unpleasant ideas \\
			\hline
			\textbf{Oxymoron} & Combination of two contradictory terms & "Jumbo shrimp" & Adds humor, irony, and unexpectedness to language \\
			\hline
			\textbf{Pun} & Play on words that have multiple meanings & "Time flies like an arrow; fruit flies like a banana" & Adds humor and wit to language \\
			\hline
			\textbf{Enumeration} & Listing words or phrases for emphasis & "I will not be pushed, filed, stamped, indexed, briefed, debriefed, or numbered!" & Adds emphasis \\
			\hline
		\end{tabular}
	\end{center}

	\section{How to write X}

	\subsection{Summary}
	\subsubsection{General aspects}
	\begin{itemize}
		\item Most relevant facts and overall meaning
		\item No own thoughts/opinions
		\item Introduction!
		\item Zero direct speech/quotations
		\item Must be factual
		\item Written in present tense
		\item No irrelevant details
		\item Closing sentence
		\item No analysis; only depiction of the text
	\end{itemize}

	\subsubsection*{Introduction}
	Your introduction sentence should answer the w-questions and inform about the source of the text (author, title, place, time, topic, excerpt/abbreviation)

	\subsubsection{Style}
	Write your summary using formal, standard english with no short forms in the present tense. Do not quote anything or use direct speech. Do not copy words/phrases, but use synonyms and paraphrase the text. 

	\subsection{Analysis}
	\subsubsection{General aspects}
	\begin{itemize}
		\item Read the assignment carefully and stick to it!
		\item Pay attention to keywords in the assignment
		\item Read the text multiple times to understand in completely
		\item Find the main message of the text
		\item Make a draft
		\item Quote your findings
		\item Pay attention to details: do not cover everything, but the most relevant aspects
		\item No personal opinion; be factual and neutral
	\end{itemize}

	\subsubsection{Structure}
	\paragraph{Introduction}
	Formulate a connecting sentence. Refer to a relevant aspect from your comprehension.
	Give a short outline of the structure of the text. Refer to the writers line of argument and the general message of the text. Do not repeat the introduction from your comprehension

	\paragraph{Main part}
	Use the three-step-method: Quote, name, explain.
	Do not only follow the order of the text, but focus on aspects relevant to the assignment (keywords!). Quote your findings correctly.

	\paragraph{Connclusion}
	Refer to the introduction and formulate a concluding sentence, for example by referring to the message or type of the text. Do not evaluate the text, but stay factual and concise.


	\subsection{Mediation}
	\subsubsection{General aspects}
	A mediation is a translation of a given text under consideration of the situation given by the task. In a mediation, you have to think about the addressee(s), the meaning of your message and the given text and cultural oder situational aspects.

	\subsubsection{How to mediate}
	\begin{itemize}
		\item Select only the msot relevant information
		\item Focus on the information your addressee(s) need
		\item No literal translation
		\item Consider the situation/cultural background of the addresse(s)
		\item Consider situational aspects (private, professional etc)
		\item Give additional information from your own knowledge if necessary
		\item Take the type of text you are supposed to write into account
		\item Use compensation strategies (i.e. paraphrasing)
	\end{itemize}

	\subsubsection{Criteria}
	\begin{itemize}
		\item Has the purpose and intention of the text been conveyed?
		\item Has the assignment been considered?
		\item Does the text meet the characteristics of the type of text?
		\item Is the formulation independent?
		\item Does the style of the text meet its intention?
	\end{itemize}

	\subsection{Email}
	\subsubsection{General aspects}
	\begin{itemize}
		\item Write an appropriate salutation
		\item Close the email with a complimentary close (i.e. yours, sincerely, kind/best regards) and your name
		\item Choose an appropriate style
	\end{itemize}

	\subsection{Comment}
	\subsubsection{General aspects}
	In a comment, you have to express your personal opinion on a topic.

	\subsubsection{Structure}
	\paragraph{Introduction}
	Raise a question or refer to a problem. Clarify your topic and/or concern.

	\paragraph{Main part}
	Follow the scheme thesis - argument - example. Name some counter arguments in order to debunk them. In contrast to a discussion, your comment is not supposed to consider multiple views equally, but instead support your own opinion.

	\paragraph{Conclusion}
	Wrap up your comment and make your point. Refer to your introduction. Summarize your most important arguments again. Name explicitly your own view.

	\subsection{Newspaper article}
	\subsubsection{General aspects}
	\begin{itemize}
		\item Write clear, concise and factual
		\item Write short paragraphs
		\item Write precise and explain technical terms
		\item Use the right tense, e.g. past tense for events that already happened
		\item Quote people and give sources 
		\item Personalize your article
	\end{itemize}

	\subsubsection{Structure}
	\paragraph{Headline}
	Try to catch the readers attention with a catchy headline
	
	\paragraph{Sub-headlines}
	Give the newspaper a structure with some informative sub-headlines

	\paragraph{First paragraph}
	Introduce the main point and the main idea of your article.

	\paragraph{Other paragraphs}
	Answer w-questions about the topic

	\subsection{Speech}
	\subsubsection{Structure}
	\paragraph{Introduction}
	\begin{itemize}
		\item Greet the audience
		\item Try to get the audiences attention (say something controversial / thought provoking / cite someone)
		\item Introduce yourself and your subject
		\item Compliment the place where you give the speech
	\end{itemize}

	\paragraph{Main part}
	\begin{itemize}
		\item Start with a thought-provoking or important argument
		\item Write progressive (problem -> solution) or antithetical (contrasting ideas)
		\item Support your arguments (facts, quotations, examples)
		\item Use some rhetorical devices
		\item Use linking words and connectors to connect your arguments and make your speech more understandable
	\end{itemize}

	\paragraph{Conclusion}
	\begin{itemize}
		\item Give a short/concise summary
		\item Finish with a call for action or similar
		\item Refer to the beginning of your speech
	\end{itemize}

	\section{General tips}
	\subsection{Make a draft outline}
	If you are not sure how to structure your text or how to begin, make a concise outline of what you want to say. Think about what type of text you are supposed to write and what needs to be in this text. Do not make your draft to detailed: it is not intended to include everything, but rather give you a short overview about what you want to write

	\subsection{Read the assignment very carefully}
	Read the assignment multiple times before starting to write. Focus on some keywords and think twice about what is expected from you. Pay attention to the \textit{Operatoren}!


\end{document}
