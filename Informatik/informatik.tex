\documentclass{article}

\usepackage{graphicx}
\usepackage{amsfonts,amsmath,amssymb,amsthm}
\usepackage{stmaryrd}
\usepackage{svg}
\usepackage{float}
\usepackage[inner=3cm, outer=3cm]{geometry}

\title{Informatik}
\date{Q2 2022/2023}
\author{Paul SK \& Sophie}

\begin{document}
	\pagenumbering{gobble}
	\maketitle
	\newpage

	\pagenumbering{arabic}

	\section{Allgemeines}

	\section{OOP mit Java}

	\section{Lineare Datenstrukturen}

	\section{Algorithmen}

	\section{Datenbanken}
	\subsection{Das Entity-Relationship-Modell}
	\subsubsection{Verschiedene Elemente}
	\paragraph{Entität}

	\begin{figure}[h!]
		\centering
		\includesvg[width=3.0cm]{entity}
		\caption{Eine Entität wird mit einem Rechteck dargestellt}
	\end{figure}	


	\paragraph{Attribut}

	\begin{figure}[h!]
		\centering
		\includesvg[width=3.0cm]{attribute}
		\caption{Ein Attribut wird in einer Ellipse geschrieben}
	\end{figure}	

	\section{Nicht-Lineare Datenstrukturen}

	\section{Formale Sprachen, Grammatiken und Automaten}
	
	\subsection{Formale Sprachen}
	
	\subsubsection{Definition}
	
	\begin{itemize}
		\item Menge von Symbolanordnungen
		\item vorgegebene Regeln
		\item Syntax: erlaubte Ausdrücke
		\item Semantik: Bedeutung von Ausdrücken (in formalen Sprachen nicht definiert)
	\end{itemize}
	
	\subsubsection{Alphabet}
	
	\begin{itemize}
		\item Menge zulässiger Zeichen für eine Sprache
		\item Bezeichnung: $\Sigma$
	\end{itemize}
	
	\subsubsection{Wort}
	
	\begin{itemize}
		\item beliebige endliche Aneinanderreihung von Zeichen des Alphabets
		\item leeres Wort ($\epsilon$): Wort ohne Buchstaben
		\item Menge aller Worte aus dem Alphabet $\Sigma$ inklusive $\epsilon$: $\Sigma$*
		\item $\Sigma$* kann Worte mit unendlicher Länge enthalten und enthält daher unendlich viele Worte
	\end{itemize}
	
	\subsubsection{Syntaxdiagramme}
	
	\begin{itemize}
		\item mehrere Symbole hintereinander
		\item Iteration (0 - $\infty$)
		\item Iteration (1 - $\infty$)
		\item Alternativen
		\item Nichts als Alternative
		\item Bezug auf andere Syntaxdiagramme
	\end{itemize}
	
	\subsection{Grammatiken}
	
	\begin{itemize}
		\item Grammatik $G = (N, \Sigma, S, P)$
		\item $N$: Menge der Nichtterminalsymbole ("Variablen")
		\item $\Sigma$: Menge der Terminalsymbole / Alphabet
		\item $S$: Startsymbol mit $S \in N$
		\item $P$: Menge der Produktionsregeln \\
		Beispiele:
		\begin{itemize}
			\item A $\rightarrow$ B
			\item A $\rightarrow$ B, B $\rightarrow$ A
			\item A $\rightarrow$ 0 $\mid$ 1
			\item A $\rightarrow$ 0B $\mid$ 100B $\mid$ $\epsilon$
		\end{itemize}
		\item $N \cap \Sigma = \varnothing$ (nichts ist gleichzeitig Terminal- und Nichtterminalsymbol)
	\end{itemize}
	
	\subsubsection{Reguläre Grammatiken}
	
	
	
	\subsection{Automaten}

	\section{Datenschutz}

\end{document}
